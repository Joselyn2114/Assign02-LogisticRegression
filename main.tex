% !TeX program = pdflatex
\documentclass[conference]{IEEEtran}
\IEEEoverridecommandlockouts

% ---- Paquetes ----
\usepackage[spanish]{babel}
\usepackage[utf8]{inputenc}
\usepackage[T1]{fontenc}
\usepackage{amsmath, amssymb}
\usepackage{booktabs}
\usepackage{siunitx}
\usepackage{graphicx}
\usepackage{array}
\usepackage{hyperref}
\usepackage{multirow}
\usepackage{caption}
\usepackage{subcaption}
\usepackage{float}
\sisetup{detect-weight=true,detect-family=true}

% Rutas comunes para figuras
\graphicspath{{figs/}{figures/}{images/}{img/}{./}}

% --- Compactación de floats/captions ---
\captionsetup{font=footnotesize}
\setlength{\textfloatsep}{6pt plus 1pt minus 2pt}
\setlength{\floatsep}{6pt plus 1pt minus 2pt}
\setlength{\intextsep}{6pt plus 1pt minus 2pt}
\setlength{\abovecaptionskip}{4pt}
\setlength{\belowcaptionskip}{0pt}

% --- Política de colocación de floats ---
\renewcommand{\topfraction}{0.9}
\renewcommand{\bottomfraction}{0.8}
\renewcommand{\textfraction}{0.07}
\renewcommand{\floatpagefraction}{0.8}
\renewcommand{\dbltopfraction}{0.9}
\renewcommand{\dblfloatpagefraction}{0.8}
\setcounter{topnumber}{5}
\setcounter{bottomnumber}{5}
\setcounter{totalnumber}{10}

% --- Utilidades de placeholders de figuras ---
\newcommand{\placeholderbox}[2][5cm]{%
  \fbox{\begin{minipage}[c][#1][c]{0.9\linewidth}\centering \textbf{Placeholder:} #2\end{minipage}}}

% label, caption, filename
\newcommand{\placefigure}[3]{%
\begin{figure}[!t]\centering
\IfFileExists{#3}{\includegraphics[width=\linewidth]{#3}}{\placeholderbox{#3}}
\caption{#2}\label{#1}\end{figure}}

% width, filename, subcaption, label
\newcommand{\placesubfig}[4]{%
\begin{subfigure}{#1}
\centering
\IfFileExists{#2}{\includegraphics[width=\linewidth]{#2}}{\placeholderbox{#2}}
\caption{#3}\label{#4}
\end{subfigure}}

% ---- Título y autores ----
\title{Implementación de Regresión Logística - PyTorch\vspace{-0.35em}}

\author{\IEEEauthorblockN{Priscilla Jim\'enez Salgado, Fabi\'an Araya Ortega, David Acu\~na L\'opez}
\IEEEauthorblockA{Curso de Inteligencia Artificial, Escuela de Ingenier\'ia en Computaci\'on, Tecnol\'ogico de Costa Rica (TEC)\\}}

\begin{document}
\raggedbottom
\maketitle

\begin{abstract}
El siguiente trabajo presenta un análisis exploratorio de datos (EDA) y la preparación del dataset Wine Quality (Vinho Verde, vino tinto) para su uso en un modelo de regresión logística. Se examinan propiedades fisicoquímicas y sensoriales, identificando \emph{outliers} y correlaciones entre variables. A partir del análisis, se aplican transformaciones logarítmicas en variables con colas largas, se seleccionan seis predictores relevantes y se redefine la calidad como variable binaria balanceada. Asimismo, se realizan divisiones estratificadas de los datos en entrenamiento, validación y prueba, y se aplica un escalado estándar. Los hallazgos enfatizan la relevancia de \texttt{alcohol}, \texttt{volatile acidity}, \texttt{sulphates} y \texttt{citric acid} como determinantes de la calidad, estableciendo una base sólida para el modelado predictivo.
\end{abstract}

\begin{IEEEkeywords}
Regresión logística, Análisis exploratorio de datos (EDA), Wine Quality Dataset, Clasificación binaria, Preprocesamiento de datos, Outliers, Machine Learning
\end{IEEEkeywords}

\section{Introducci\'on}
Se desarrolló un flujo reproducible para preparar el conjunto de datos previo al modelado: lectura y verificación de estructura, limpieza de duplicados, análisis exploratorio de datos (distribuciones, \emph{outliers}, correlaciones) y particionado estratificado con estandarización. El objetivo es dejar un dataset depurado, balanceado y escalado que permita una primera aproximación con regresión logística y sirva de línea base para futuros modelos.

\section{Conjunto de Datos y Limpieza}
El dataset contiene \num{1599} filas y \num{12} columnas: once variables fisicoquímicas continuas y la variable \texttt{quality} (entera). Todas las columnas tienen 1599 valores no nulos y tipos consistentes (\texttt{float64} para predictores y \texttt{int64} para \texttt{quality}). Se identificaron \textbf{240 duplicados exactos}; para evitar sesgo en estimación de parámetros se optó por \emph{eliminarlos} con \texttt{drop\_duplicates()}.

\placefigure{fig:info}{Estructura del dataset: filas/columnas, tipos y no-nulos.}{preview_info.png}

\subsection{Resumen estructural}
\begin{table}[H]\centering\caption{Características generales del conjunto de datos}
\label{tab:estructura}
\begin{tabular}{lS}
\toprule
\textbf{Caracter\'istica} & {\textbf{Valor}}\\
\midrule
Filas & 1599\\
Columnas & 12\\
Tipos & \texttt{float} y \texttt{int} \\
Faltantes & 0 en todas las columnas \\
Duplicados exactos & 240\\
\bottomrule
\end{tabular}
\end{table}

\section{An\'alisis Exploratorio de Datos (EDA)}
\subsection{Estad\'isticas y distribuciones}
No hay faltantes. Las distribuciones muestran colas derechas y valores extremos plausibles en \texttt{residual sugar}, \texttt{chlorides}, \texttt{total sulfur dioxide} y \texttt{sulphates}. Se propone \textbf{transformación logarítmica} (o winsorización ligera) para estabilizar varianza sin perder información.

\placefigure{fig:desc}{Tabla resumen con medias, desviaciones y percentiles por variable.}{describe_table.png}

\placefigure{fig:box}{Boxplots de variables con colas largas y \emph{outliers} plausibles.}{eda_boxplots.png}

\subsection{Correlaciones y coherencia fisicoqu\'imica}
Se observan patrones esperados: relación negativa \texttt{alcohol}–\texttt{density} y positiva \texttt{residual sugar}–\texttt{density}. Con \texttt{quality} destaca señal positiva de \texttt{alcohol}, \texttt{sulphates} y \texttt{citric acid}; negativa de \texttt{volatile acidity}, \texttt{density} y \texttt{chlorides}. Para reducir colinealidad se prioriza \texttt{total sulfur dioxide} sobre \texttt{free sulfur dioxide}.

\begin{figure}[!t]\centering
\placesubfig{0.48\linewidth}{heatmap_pearson.png}{Matriz de correlaci\'on de Pearson}{sub:pear}
\hfill
\placesubfig{0.48\linewidth}{heatmap_spearman.png}{Matriz de correlaci\'on de Spearman}{sub:spear}
\caption{Mapas de calor de correlaciones y posibles colinealidades.}
\label{fig:heatmaps}
\end{figure}

\placefigure{fig:scatters}{Relaciones bivariadas clave entre variables fisicoquímicas y calidad.}{scatters_grid.png}

\subsection{Candidatos a predictores}
Del EDA surgen como candidatas principales \texttt{alcohol}, \texttt{volatile acidity}, \texttt{sulphates} y \texttt{citric acid}. Con transformaciones logarítmicas se incorporan además \texttt{total\_sulfur\_dioxide\_log} y \texttt{chlorides\_log} para un máximo de seis \emph{features} informativas y con menor colinealidad.

\begin{table}[!t]\centering
\caption{Selección final de \emph{features} y justificación}
\label{tab:featimp}
\setlength{\tabcolsep}{4pt}
\renewcommand{\arraystretch}{1.1}
\footnotesize
\begin{tabular}{@{}>{\raggedright\ttfamily}p{0.36\linewidth} >{\raggedright\arraybackslash}p{0.58\linewidth}@{}}
\toprule
\textbf{Feature} & \textbf{Motivación breve}\\
\midrule
alcohol & Señal positiva con calidad; coherencia fisicoquímica con densidad.\\
volatile\_acidity & Señal negativa (aroma avinado) consistente en EDA.\\
sulphates\_log & Asociación positiva; la transformación log reduce cola derecha.\\
citric\_acid & Mejora perfil sensorial; correlación moderada.\\
total\_sulfur\_dioxide\_log & Sustituye a \texttt{free SO$_2$}; menor colinealidad.\\
chlorides\_log & Salinidad alta penaliza calidad; el log mitiga outliers.\\
\bottomrule
\end{tabular}
\end{table}

\section{Divisi\'on del Conjunto de Datos}
\subsection{Variable objetivo binaria}
Se define \texttt{quality\_label} = 0 (MALA) si \texttt{quality}\,$\leq$\,5 y 1 (BUENA) si \texttt{quality}\,$\geq$\,6. La distribución resultante es: clase 1 (BUENA) = 719, clase 0 (MALA) = 640. El balance es moderado.

\placefigure{fig:classdist}{Distribuci\'on de clases en la variable objetivo binaria.}{class_distribution.png}

\subsection{Transformaciones y estandarizaci\'on}
Se crean columnas con sufijo \texttt{\_log} para variables con colas largas: \texttt{residual\_sugar\_log}, \texttt{chlorides\_log}, \texttt{total\_sulfur\_dioxide\_log}, \texttt{sulphates\_log}. Las \emph{features} finales (máximo seis) son:
\begin{itemize}
\item \texttt{alcohol}, \texttt{volatile\_acidity}, \texttt{sulphates\_log},
\texttt{citric\_acid}, \texttt{total\_sulfur\_dioxide\_log}, \texttt{chlorides\_log}.
\end{itemize}
El escalado estándar se ajusta solo con \emph{Train} y se aplica a \emph{Val/Test} para evitar fuga de información.

\subsection{Split estratificado}
Se emplea \emph{stratified sampling} para mantener proporciones de clase en todos los subconjuntos. Tamaños: \textbf{Train: 951}, \textbf{Validation: 204}, \textbf{Test: 204}. Las proporciones por clase en cada subconjunto son $\approx$ 0.53 (BUENA) / 0.47 (MALA).

\section{Bibliograf\'ia}

\renewcommand{\refname}{}
\vspace{-1em}%
\begin{thebibliography}{00}

\bibitem{CienciadedatosCorr}
``Correlaci\'on lineal con python,'' \emph{Ciencia de Datos}. [En l\'inea]. Disponible en: \url{https://cienciadedatos.net/documentos/pystats05-correlacion-lineal-python}

\bibitem{SklearnUG}
scikit\mbox{-}learn developers, ``User Guide,'' 2007--. Disponible en: \url{https://scikit-learn.org/stable/user_guide.html}

\bibitem{SeabornTut}
Michael Waskom, ``Seaborn Tutorial,'' 2012--. Disponible en: \url{https://seaborn.pydata.org/tutorial.html}

\bibitem{PandasUG}
NumFOCUS, Inc., ``Pandas User Guide,'' 2024--. Disponible en: \url{https://pandas.pydata.org/pandas-docs/stable/user_guide/index.html}

\bibitem{NumpyMatmul}
NumPy Developers, ``numpy.matmul --- NumPy v2.2 Manual.'' Disponible en: \url{https://numpy.org/doc/stable/reference/generated/numpy.matmul.html}

\bibitem{Kaggle_Dans}
Dan Becker, ``Using Categorical Data with One Hot Encoding,'' \emph{Kaggle}, 2018. Disponible en: \url{https://www.kaggle.com/code/dansbecker/using-categorical-data-with-one-hot-encoding}

\bibitem{CastrilloOutliers}
Marta Castrillo, ``C\'omo identificar y tratar outliers con Python,'' \emph{Medium}, 2023. Disponible en: \url{https://medium.com/@martacasdelg/c%C3%B3mo-identificar-y-tratar-outliers-con-python-bf7dd530fc3}

\end{thebibliography}

\end{document}
